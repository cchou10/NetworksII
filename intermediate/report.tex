\documentclass[11pt,a4paper]{article}
%\usepackage{fullpage}
\usepackage[total={6.5in,8.75in},
top=1.2in, left=0.9in]{geometry}
\usepackage[utf8]{inputenc}
\usepackage[english]{babel}
\usepackage{amsmath, amsthm}
\usepackage{amsfonts}
\usepackage{amssymb}
\usepackage{graphicx}
\usepackage{pifont}
%\usepackage{bussproofs}
\usepackage{enumitem}
\usepackage{centernot}
\usepackage{xspace}
\usepackage{stmaryrd}
\usepackage{mathtools}
\usepackage{cleveref}
\usepackage{listings}
\usepackage{tikz-qtree}
\usepackage{tikz}
\usepackage{algorithm}
\usepackage{algorithmicx}
\usepackage{algpseudocode}
\usepackage[hidelinks]{hyperref}
\usetikzlibrary{automata,trees,fit,backgrounds,shapes,snakes}
\usetikzlibrary{decorations.shapes}
\usepackage{float}
\usepackage{fancyvrb}
\usepackage{framed}
\usepackage{fancyhdr}
\usepackage{lastpage}
\usepackage{comment}

\usepackage[backend=bibtex,
%style=numeric,
style=alphabetic,
%style=reading,
sorting=ynt
]{biblatex}
\addbibresource{references}

\definecolor{light-gray}{gray}{0.95}
\newcommand {\conf} [1] {\ensuremath{\left\langle #1 \right\rangle}}
\newcommand {\bstep} {\ensuremath{\Downarrow}}
\newcommand {\bstepA} {\bstep_{ A}}
\newcommand {\bstepB} {\bstep_{ B}}
\newcommand {\bstepC} {\bstep_{ C}}
%\newcommand {\co} [1] {\ensuremath{\operatorname{\bf #1}}}
\newcommand {\coo} [1] {\ensuremath{\operatorname{\mathsf{#1}}}}
\newcommand {\co} [1] {\coo{#1}}
\newcommand {\pp}  {\ensuremath{\mbox{\footnotesize{++}}}}
\newcommand {\Skip} {\co{skip}}
\newcommand {\Not} {\co{not}}
\newcommand {\Iff}[3] {\co{if} (#1) \co{then} #2 \co{else} #3}
\newcommand {\Ifp}[3] {\co{ifp} (#1) \co{then} #2 \co{else} #3}
%\newcommand {\While}[2] {\co{while} #1 \co{do} #2}
%\newcommand {\Repeat}[2] {\co{repeat} #1 \co{do} #2}
\newcommand {\Input} {\co{input}}
%\newcommand {\Break} {\co{break}}
%\newcommand {\Continue} {\co{continue}}
\newcommand{\True}{\co{True}}
\newcommand{\False}{\co{False}}
\newcommand{\Or}{\co{or}}
\newcommand{\Let}[1]{\coo{let} #1 \coo{in} }
\newcommand{\Lam}{\ensuremath{{\lambda}}}
\newcommand{\Ref}{\coo{ref}}
\newcommand{\Int}{\coo{int}}
\newcommand{\bool}{\coo{bool}}
\newcommand{\Bool}{\bool}
\newcommand{\Unit}{\coo{unit}}
\newcommand{\Rec}[1]{\left\{#1\right\}}
\newcommand{\pa}[1]{\left(#1\right)}
\newcommand{\dt}[1]{\left|\arr{#1}\right|}
\newcommand{\ba}[1]{\left\langle #1\right\rangle}
\newcommand{\tree}{\coo{tree}}
\newcommand{\fa}{\coo{\forall}}
\newcommand{\more}[1]{\vdots\hspace{-1mm}~^{#1}}
\newcommand{\f}[1]{\textsc{#1}}
\newcommand{\g}[1]{\textsf{#1}}
\newcommand{\finite}{\co{finite}}
\newcommand{\lift}[1]{\left\lfloor #1 \right\rfloor}
\newcommand{\ttwo}{\mbox{\scriptsize\ding{173}}}

\newcommand{\trans}[2]{\ensuremath{\mathcal{#1}\left\llbracket #2\right\rrbracket}}

\newtheorem*{lemma}{Lemma}
\newtheorem*{theorem}{Theorem}
\newtheorem*{definition}{Definition}
\newtheorem*{corollary}{Corollary}

\lstset{ %
  language=Matlab,                % the language of the code
  basicstyle=\footnotesize,           % the size of the fonts that are used for the code
  numbers=left,                   % where to put the line-numbers
  numberstyle=\tiny\color{gray},  % the style that is used for the line-numbers
  stepnumber=1,                   % the step between two line-numbers. If it's 1, each line 
                                  % will be numbered
  numbersep=10pt,                  % how far the line-numbers are from the code
  backgroundcolor=\color{white},      % choose the background color. You must add \usepackage{color}
  showspaces=false,               % show spaces adding particular underscores
  showstringspaces=false,         % underline spaces within strings
  showtabs=false,                 % show tabs within strings adding particular underscores
  mathescape=true,
  frame=leftline,                   % adds a frame around the code
  rulecolor=\color{gray},        % if not set, the frame-color may be changed on line-breaks within not-black text (e.g. comments (green here))
  tabsize=2,                      % sets default tabsize to 2 spaces
  captionpos=t,                   % sets the caption-position to bottom
  breaklines=true,                % sets automatic line breaking
  breakatwhitespace=false,        % sets if automatic breaks should only happen at whitespace
  title=\lstname,                   % show the filename of files included with \lstinputlisting;
                                  % also try caption instead of title
  escapeinside={\%*}{*)},            % if you want to add LaTeX within your code
  morekeywords={*,...},              % if you want to add more keywords to the set
  deletekeywords={...}              % if you want to delete keywords from the given language
}

\newcommand\tmark[2]{%
  \ensuremath{\tikz[baseline] \node[anchor=base] (#1) {#2};}}
\tikzstyle{every picture}+=[remember picture]
\newcommand{\tm}[2]{\tmark{#1}{\ensuremath{#2}}}
\newcommand{\tr}[2]{\tmark{#1}{\color{red}{\ensuremath{#2}}}}
\newcommand{\arr}[1]{\begin{array}{cccccccccc} #1\end{array}}
\newcommand{\mat}[1]{\left(\arr{#1}\right)}
\newcommand{\Malloc}{\co{malloc}}
\newcommand{\Null}{\co{null}}

\newcommand{\SN}{\ensuremath{\mathcal{SN}}}
\newcommand{\Inl}{\co{inl}}
\newcommand{\Inr}{\co{inr}}
\newcommand{\Case}[3]{\co{case}~#1~\co{of} #2 \mid #3}
\newcommand{\R}{\co{rec}}
\newcommand{\Fold}{\co{fold}}
\newcommand{\Unfold}{\co{unfold}}
\newcommand{\coerce}[1]{\ensuremath{\Theta\left\llbracket #1\right\rrbracket}}

\allowdisplaybreaks

\usepackage{setspace}
\doublespacing

\setlength{\headheight}{15pt}
 
\pagestyle{fancyplain}
%\renewcommand{\chaptermark}[1]{\markboth{#1}{}}
 
\lhead{\fancyplain{}{\footnotesize\leftmark}}
\chead{}
\rhead{\fancyplain{}{\textsc{Lee Gao}}}
\lfoot{}
\cfoot{\thepage\ of \pageref{LastPage}}
\rfoot{}

\usepackage[protrusion=true,expansion=true]{microtype} % Better typography
\usepackage{graphicx} % Required for including pictures
\usepackage{wrapfig} % Allows in-line images

\author{Clifford Chou, Bryan Cuccioli, Lee Gao, 
Favian Contreras}
\date{\today}
\title{\textbf{Intermediate Project Report}} % Subtitle

\begin{document}
\tikzset{every tree node/.style={minimum width=2em,draw,circle},
         blank/.style={draw=none},
         edge from parent/.style=
         {draw, edge from parent path={(\tikzparentnode) -- (\tikzchildnode)}},
         level distance=1.5cm}


\renewcommand{\abstractname}{Abstract} % Uncomment to change the name of the abstract to something else
\begin{singlespace}
\maketitle
\begin{abstract}
We explore what we believe to be a novel approximate measure in the ``efficiency'' of large software engineering projects. Many software systems are so large and complex that it becomes impossible to reason about the runtime properties of such systems using conventional techniques in runtime analysis. As such, an approximation of this property is often used, but even then, it is difficult to create heuristics to judge whether a system is efficient or not. Our approach treats the interaction between the different subroutines of a system as network behaviors and approximates the running time of a system by the most number of calls to a single function. Having experimentally determined that various large software systems behave like scale-free networks, we gain an asymptotic characterization of the growth of the max-degree node within the call-graph which lets us approximate that the running time of such systems with code size $n$ slows down according to $O(\sqrt[\alpha]{n})$ for some $1 < \alpha \le 2$.
\end{abstract}

\hspace*{3,6mm}\textit{Keywords:} {\sf \small  scale-free networks, software engineering, static analysis, linear interpolation} % Keywords
\end{singlespace}
\vspace{30pt} % Some vertical space between the abstract and first section

%\setlength{\parindent}{0pt}

One of the biggest problems in the software development world is trying to reason about the efficiency of the software systems that are being developed. Traditional runtime analysis techniques tend to focus on algorithms as individual components and is rarely amenable to large scale systems because it's inherently difficult to explore the relationship between large numbers of different components of the system. Therefore, there's a demand to automatically analyze these systems.

Conventional techniques are rather cumbersome and requires a lot of sophisticated machinery. Oftentimes, these techniques do not scale to large projects with complex interactions between their various subroutines. However, it seems from experience profiling and optimizing code that most of the benefits come from reducing either the code size or the complexity of functions that get called the most. As such, we devised a measure of the complexity of a project based on how many incoming function calls each function might receive. One motivation for this is because this measure essentially maps to the indegree $k_{max}$ parameter (the expected maximum degree of the nodes of a network) of a graph structure that models the calls-into interaction between the various functions, what we will from now on call a call-graph.

More formally, using a simplified model, let's define a program $P$ as a collection of functions $\ba{\vec f}$ along with a main function $f^* \in \ba{\vec f}$ acting as the entry-point. Define the function $\f{calls-to}: \g{fun} \to \g{set}(\g{fun})$ that returns the set of functions that \textit{can}\footnote{This is actually a really tricky notion, since the function $f \triangleq \lambda x: \Iff{1 = 2}{g(x)}{h(x)}$ \textbf{cannot} in actuality call $g$ because it's never the case that $1 = 2$, but since this happens so rare in practice, we will say that $f$ can call $\Rec{g,h}$} be called. Furthermore, let's define a relation $\co{calls} \subseteq \g{fun} \times \g{fun}$ such that $$f \co{calls} g \iff g \in \f{calls-to}(f).$$
Define the call-graph induced by program $P$ to be 
$$G_P = \pa{\ba{\vec f}, \co{calls}}$$
such that the nodes of $G_P$ are the functions in the program and the edges are the pairs $(f,g) \in \co{calls}$\footnote{Recall notationally, $f\co{calls}g$ is equivalent to $(f,g) \in \co{calls}$} so that we connect an edge from $f$ to $g$ if $f$ can call $g$.

Because programming languages obey a very strict semantic model, we hypothesized that this likely means that there's some kind of preferential attachment to certain functions over others. As such, we believed that $G_P$ will be scale-free in the sense that\cite{DUR}
\begin{itemize}
\item The indegree distribution of the nodes of $G_P$ forms a powerlaw distribution; suppose $p(k)$ is the probability that a vertex in $G_P$ has indegree $k$, then $p(k) \sim C k^{-\gamma}$ and
\item furthermore, $2 \le \gamma \le 3$.
\end{itemize}
It turns out that scale-free networks have various useful properties that we can exploit, one of which is the characterization of the expected maximum indegree of the graph $k_{max}$, which corresponds to the function in $G_P$ that gets \emph{called into} the most often. 
\section*{Characterizing $k_{max}$ for Scale-Free Networks}
Recall that the indegree distribution of a scale-free network has probability density function
$$
p(k) = C k^{-\gamma}
$$
where 
$$
C \sum_{k=0}^\infty k^{-\gamma} = 1 \implies C = \frac{1}{\sum\limits_{k=0}^\infty k^{-\gamma}} = \zeta(\gamma)^{-1}
$$
if we approximate $\sum \approx \int$, then
$$
\zeta(\gamma) \approx \int_1^\infty x^{-\gamma} dx = \pa{\gamma-1}^{-1}
$$
Now, suppose we want to find $k_{max}$, then in a discrete sense, we would expect that the probability of $p(k > k_{max})$ must be smaller than the probability to just pick one node out of all of the $n$ nodes, so that, working in the continuous approximation
$$
\int_{k_{max}}^\infty (\gamma -1)k^{-\gamma} dk = k_{max}^{1-\gamma} \approx \frac{1}{n}
$$
and so $k_{max} \approx \sqrt[\gamma - 1]{n}$. \cite{CLASS}

\medskip

\printbibliography[title={References}]

\end{document}
