\documentclass[11pt,a4paper,twocolumn]{article}
\usepackage{paper}

\author{Clifford Chou, Favian Contreras, Bryan Cuccioli, Lee Gao}
\title{Scale-Free Networks in Static Call Graphs}
\date{\today}

% We only want this in the final report, not the intermedidate report.
\setlength{\jot}{.3cm}
\makeatletter
\renewcommand*\env@matrix[1][\arraystretch]{%
  \edef\arraystretch{#1}%
  \hskip -\arraycolsep
  \let\@ifnextchar\new@ifnextchar
  \array{*\c@MaxMatrixCols c}}
\makeatother

\begin{document}
\begin{singlespace}

\maketitle

\begin{abstract}
{We explore a novel approximate measure in the efficiency of large-scale
software engineering projects. Many software systems are so large and complex
that it is impossible to reason about their runtime properties using
conventional techniques in runtime analysis. Our approach treats the
interaction between the different subroutines of a system as network behaviors
and approximates the running time via the distribution of static function
calls.}

{Having experimentally determined that various large software systems
behave like scale-free networks, we gain an asymptotic characterization of the
growth of the max-degree node within the call-graph, which lets us approximate
 the running time of such systems with code size $n$ according
to $O\left(\sqrt[\alpha]{n}\right)$ for some $\alpha \approx 1.25$.}
% TODO(Favian): Add specific information about experimental results.
\end{abstract}

\hspace*{3,6mm}\textit{Keywords:} {\sf \small scale-free networks, software
engineering, static analysis, linear interpolation}
\vspace{10pt}

\section{Introduction}

A big problem in software engineering is attempting to reason about
the efficiency of the software system being developed. Traditional runtime
analysis techniques tend to focus on algorithms as individual components.
Because of the inherent difficulty in exploring the relationship between
large numbers of different components of the system, such techniques are
rarely amenable to large-scale projects.

Traditional techniques are rather cumbersome to wield and require a lot of
sophisticated machinery and analysis. 
Oftentimes, these techniques do not scale to large
projects with complex caller-callee relationships between their various subroutines. 
However, conventional wisdom and anecdotal experience in the profiling and 
optimization of software code indicate
that most of the benefits come from reducing the complexity of
the functions that get called most often. %As such, we devise a measure of the
complexity of a project based on how many incoming function calls each
function might receive.

Using this rule of thumb, one might then view the complexity of the caller-callee
relationship of each function as a statistical measure of the the runtime of the 
software been developed. For one thing, if a subroutine $f$ is called by a million 
different functions, then by reducing the overall runtime of $f$ by a small factor 
might net us a million-fold increase in efficiency. Indeed, it has been observed
time and time again \cite{something} that the most-called function of a software 
is typically the bottle neck in the software runtime. Therefore, a natural idea for
measuring the complexity and the bloat of a large software project is to measure 
the expected \emph{most} number of calls that any function within the software 
will receive.

While this expected maximum number of called function is somewhat difficult/tedious
to compute (especially for large projects), we found that it is possible to accurately
estimate this statistic using only the number of functions alone. We introduce a 
network modeling the caller-callee interaction between various subroutines known
as a call-graph and posit that all such graph structures exhibit powerlaw degree 
distribution. Using existing analysis on such networks, we develop an analytic 
model describing the growth of the runtime complexity of software based on the
number of subroutines.

%\section{Related Work}
% TODO(Lee): Add a couple paragraphs about existing related work.

\section{Mathematical Model}
\subsection{Viewing a Program as a Network}
One motivation for the aforementioned measure is that it maps to the in-degree
$k_{max}$ parameter -- the expected maximum degree of the nodes of a network --
of a graph structure that models the calls-into interaction between the various
functions, which we will henceforth refer to as a \emph{call-graph}.

More formally, using a simplified model, define a program $P$ as a collection
of functions $\ba{\vec{f}}$ along with a \emph{main function}
$f^{\ast}\in\ba{\vec{f}}$ acting as the entry-point. Define the function
\[\f{calls-to}:\g{fun}\to\g{set}(\g{fun})\]
which returns the set of functions that \emph{can} be called.\footnote{This is
a subtle notion, since the function $f\triangleq\lambda x: \Iff{\False}{g(x)}
{h(x)}$ cannot in actuality call $g$, but since this occurs so rarely in
practice, we can approximate that $f$ can call $\{g,h\}$.} Furthermore, define
a relation $\co{calls}\subseteq\g{fun}\times\g{fun}$ such that
\[f\co{calls} g \iff g \in \f{calls-to}(f).\]
The \emph{call-graph induced by program} $P$ is then
\[G_P=\pa{\ba{\vec{f}},\co{calls}},\]
i.e. the nodes of $G_P$ represent the functions of the program $P$ and the
edges are the pairs $(f,g)\in\co{calls}$, that is, $f\co{calls} g$, so that
we can connect an edge from $f$ to $g$ if $f$ can call $g$.

Because programming languages obey a strict semantic model, we hypothesize that
there exists a preferential attachment to certain functions over others.
Consequently, we believe that $G_P$ will be scale-free in the sense that
the in-degree distribution of the nodes of $G_P$ forms a power law
distribution \cite{DUR}:
\begin{itemize}
\item suppose $p(k)$ is the probability of randomly encountering a vertex in $G_P$
with indegree $k$; then $p(k)\sim Ck^{-\gamma}$ and
\item furthermore, $2\leq\gamma\leq 3$.
\end{itemize}

It turns out that scale-free networks have various useful properties that we
can exploit, one of which is the characterization of the expected maximum
in-degree $k_{max}$ of the graph, which corresponds to the function in $G_P$
which gets \emph{called into} the most often.
\subsection{Characterizing $k_{max}$ for Scale-Free Networks}

Recall that the in-degree distribution of a scale-free network has probability
density function $p(k)=Ck^{-\gamma}$, where
\[\sum_{k=0}^{\infty} p(k) = C\sum_{k=0}^{\infty} k^{-\gamma} = 1.\]
That is,
\begin{equation}C=\left(\sum_{k=0}^{\infty} k^{-\gamma}\right)^{-1}
=\zeta(\gamma)^{-1}.\end{equation}
Using the continuous approximation, we have
\begin{equation}\zeta(\gamma)\approx \int_1^{\infty} x^{-\gamma} dx
=(\gamma-1)^{-1}.\end{equation}

In a discrete sense, we expect that the probability $p\left(k>k_{max}\right)$
must be smaller than the probability of choosing just one node from the set
of $n$ nodes ($\frac{1}{n}$), so that, working in the continuous approximation,
\begin{equation}\int_{k_{max}}^{\infty} (\gamma-1)k^{-\gamma} dk
=k_{max}^{1-\gamma}\approx \frac{1}{n}.\end{equation}
Hence $k_{\max}\approx \sqrt[\gamma - 1]{n}$ asymptotically in $n$ \cite{CLASS}.

In essence, if we can confirm that call-graphs are typically scale-free, then we know
that the maximally called function with $k_{max}$ calls grow according to 
$\sqrt[\gamma - 1]{n}$ for some constant parameter $\gamma$ in $n$ functions.
\section{Call-graphs are Scale-Free}
\subsection{Static Analysis Framework}

In order to test our hypothesis, we first needed to create the tools required
to construct $G_P$. We chose to analyze Java programs because of the large
quantity of open source software that we can use as well as the existing
support for doing this type of analysis in Java.

Redistributed Java software is typically composed of a collection of
\emph{bytecode} which can be executed by the Java Virtual Machine. Because the
bytecode language is much more structured and easier to analyze than the
textual Java programming language itself, we chose to construct the call-graph
of the translated bytecode. This also allows us to avoid issues such as
parsing and type inference which are handled by the Java compiler. In order
to parse and analyze the compiled bytecode, we used the OW2 ASM library
\cite{ocwasm}.

\begin{algorithm}
\caption{Constructing the Call Graph}
\begin{algorithmic}
\Function{Analyze}{$P = \ba{\vec f}$}
\State $G_P \gets \g{new}~ \g{Graph}()$
\For{$\g{fun}~f \in P$}
\For{$\g{instruction}~i \in f.\g{instructions}$}
\If{$i$ is a method-call to $\g{fun}~g$}
\State Add edge $f \to g$ into $G_P$
\EndIf
\EndFor
\EndFor
\EndFunction
\end{algorithmic}
\label{fig:construction}
\end{algorithm}

The static analysis of the compiled bytecode is not difficult, since we are
only interested in the invocation relationship between different functions. We
first construct the call-graph $G_P$ according to the algorithm in Algorithm
~\ref{fig:construction}. We then compute the in-degree frequency of $G_P$,
which counts the number of edges coming into each function in $G_P$. The
abstract notion of a frequency table is represented concretely as a mapping
$\f{FreqIn}:\g{fun}\to\mathbb{N}$ within our analysis. We then compute the
distribution of the degrees also concretely as a mapping $\f{Dist}:\mathbb{N}
\to\mathbb{N}$ which maps each positive degree $d$ to the number of nodes of
degree $d$ within $G_P$.

\subsection{Assessing Goodness of Fit and Linear Interpolation}

We analyzed the source code of seven large open source Java projects, including
Google Guava and ANTLR, and found that the degree distribution in each follows
a power law based on their \g{loglog} plots. Recall that if a distribution has
density $p(k)\sim Ck^{-\gamma}$, then $\log(p(k))=\log(C)-\gamma\log(k)$, so
its distribution should look roughly linear.

In order to determine the power law coefficient $\gamma$, we turn to linear
algebra. We have a list of degrees $k$ and a list of degree frequencies $y$,
and we want to find the parameters $C,\gamma$ so that $\hat{y}(k)=Ck^{-\gamma}$
best fits the observed data. Since non-linear interpolation is a hard problem,
we can transform the data logarithmically into the linear interpolation
problem
\begin{align}
\label{eq:linint}
\arg\min_{C,\gamma} \|(\log(C)-\gamma\log(k))-\log(y)\|_2^2.
\end{align}
A monotonicity argument with respect to $\log$ guarantees that the minimum
of this system is also optimal when we transform this instance back to the
non-linear version via $\exp$.

% TODO(blc) fix equation refs
Now, let $\bar{k}_i=\log k_i$, $\bar{y}_i=\log y_i$,
$\bar{C}=\log C$, $n=\g{length}(k)$, and let $A$ be the $n\times 2$ matrix
\[A=\begin{pmatrix}1&&\bar{k}_1\\ 1&&\bar{k}_2\\ &\vdots&\\ 1&&\bar{k}_n
\end{pmatrix}.\]
Then our linear interpolation problem~\eqref{eq:linint} is equivalent to finding
$x=\left(\bar{C},-\gamma\right)^T$ such that $\min_x \|Ax-\bar{y}\|$ is
minimized. We can solve this via the normal equation \cite{mcomp} so that
$x=\left(A^T A\right)^{-1} A^T\bar{y}$. That is,
\begin{align*}x&=\begin{pmatrix}[2]n&\sum \bar{k}\\
\sum \bar{k} & \bar{k}^T\bar{k}\end{pmatrix}^{-1}
\begin{pmatrix}[2]\sum \bar{y}\\ \bar{k}^T\bar{y}\end{pmatrix}\\
&=\frac{
\begin{pmatrix}[2]\bar{k}^T\bar{k}\cdot\sum\bar{y}-\sum\bar{k}\cdot
\bar{k}^T\bar{y}\\ n\bar{k}^T\bar{y}-\left(\sum \bar{k}\right)
\left(\sum\bar{y}\right)\end{pmatrix}
}{n\cdot\bar{k}^T\bar{k}-\left(\sum \bar{k}\right)^2}.\end{align*}
Thus we get a power law coefficient
\begin{equation} \label{eq:gamma}
\gamma=\frac{\left(\sum\bar{k}\right)\left(\sum \bar{y}\right)
-n\bar{k}^T\bar{y}}{n\bar{k}^T\bar{k}-\left(\sum \bar{k}\right)^2},
\end{equation}
which is easily computable.

% TODO(Lee) explain why scale-free networks are what's good
% TODO cite Guava and other projects we directly reference
In the case of Google Guava, the distribution has coefficient $\gamma\approx
2.7$, which suggests that it is within the realm of a scale-free network
\cite{CLASS}.

% TODO(Lee) talk about fat tails and cdfs

% TODO(Favian) throw in graphs and references to the rest of the data
% and talk about results

\printbibliography[title={References}]

\end{singlespace}
\end{document}
